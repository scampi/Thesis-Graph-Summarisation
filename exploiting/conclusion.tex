\section{Conclusion}

We have presented in this chapter various applications of the graph summary and of the MF ranking model. We introduced a RDF modeling of a graph summary; this allows a summary to hold the function of a schema for the entity graph. Then we have described an application of the summary which provides recommendations to a user writing a SPARQL query. These recommendations are dependent on the current state of the query, and so only contain elements that may be used at a specific position in the SPARQL query. In addition, we discussed how the MF ranking model can be used in the latter application in order to improve the ranking of recommendations. Finally, we presented another application of graph summaries with the Web Data Inspected, which allows users to delve into a dataset in order to better understand its structure; it can further benefits datasets owners by highlighting incorrect or suspicious parts of the graph.\\

As a future work, we plan to investigate the use of a summary for the purpose of optimising query execution by leveraging the statistics available with the summary. In addition, we will study the application of a summary to improve data partitioning, by relying on the knowledge of the graph's structure. Furthermore, we want to improve the Web Data Inspector so that it can better assess the quality of a dataset.
