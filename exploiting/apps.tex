\chapter{Graph Summary Applications}
\label{chap:system}

Graph summarisation allows to generate a much smaller structural replica of an entity graph. Therefore, it can benefit applications that require some knowledge about the structure of the graph. We present in this chapter some applications that are made possible thanks to the graph summary introduced in Section~\ref{chap:summary}.

We introduce in Section~\ref{chap03:sec:wd-mgnt} an RDF model for the output of a graph summarisation process. The existence of such a model enables querying the summary in order to inspect the entity structure of the graph. Combined with the entity graph, the RDFication of the summary ultimately holds the function of a \emph{schema} for the graph. In Section~\ref{sec:exploiting:sparqled:recommendation} we present an application that leverage the RDF model in order to provide RDF terms recommendation to a user writing a SPARQL query. The graph summary is also used in the Web Data Inspector described in Section~\ref{sec:web-data-inspector} as a mean for inspecting edges in a dataset and relationships between datasets.
