\chapter{Data Tables of the DAG Ranking Evaluation}

\section{Entity Ranking}

Table~\ref{tab:ranking-cmp-results} reports the evaluation results of comparing MF extensions BM25MF and PL2MF against several state-of-the-art ranking functions. The $p$-Value is computed with the two-tailed Wilcoxon matched-pairs signed-ranks test~\cite{sheskin:2003:CRC,buttcher:2010:IRI:1869919}, where a statistically significant difference at level $0.10$ is marked with one star $*$ and at level $0.05$ with two stars $**$. BM25MF and PL2MF are used as a baseline in this test. $\Delta\%$ indicates the difference in percentage between the two MAP values compared in that test.

\begin{table}
	\centering
	\ra{0.5}
	\resizebox{\textwidth}{!}{
		\begin{tabular}{lc@{\hs}rrccc@{\hs}rrccc@{\hs}rrcc}
			\toprule
			& \phantom{a} & \multicolumn{4}{c}{INEX09}
			& \phantom{a} & \multicolumn{4}{c}{SS10}
			& \phantom{a} & \multicolumn{4}{c}{SS11} \\
			& \phantom{a} & MAP & P@10 & $p$-Value & $\Delta\%$
			& \phantom{a} & MAP & P@10 & $p$-Value & $\Delta\%$
			& \phantom{a} & MAP & P@10 & $p$-Value & $\Delta\%$ \\
			\cmidrule{3-6} \cmidrule{8-11} \cmidrule{13-16}
			BM25MF & \phantom{a} & 0.1593 & 0.3982 & - & -
			& \phantom{a} & 0.1303 & 0.3783 & - & -
			& \phantom{a} & 0.1811 & 0.1880 & - & - \\
			TF-IDF & \phantom{a} & 0.1246 & 0.3109 & $3.2e^{-04**}$ & $-21.78$
			& \phantom{a} & 0.0581 & 0.2304 & $2.1e^{-10**}$ & $-55.41$
			& \phantom{a} & 0.0655 & 0.1040 & $1.4e^{-06**}$ & $-176.49$ \\
			BM25 & \phantom{a} & 0.1330 & 0.3309 & $8.4e^{-05**}$ & $-16.51$
			& \phantom{a} & 0.1350 & 0.4000 & $3.7e^{-01}$ & -
			& \phantom{a} &0.1625 & 0.1920 & $1.4e^{-01*}$ & $-11.45$ \\
			BM25F & \phantom{a} & 0.1489 & 0.3764 & $5.7e^{-03**}$ & $-6.53$
			& \phantom{a} & 0.1100 & 0.3283 & $3.0e^{-05**}$ & $-15.58$
			& \phantom{a} & 0.1401 & 0.1680 & $1.1e^{-04**}$ & $-29.26$ \\
			\midrule
			PL2MF & \phantom{a} & 0.1525 & 0.3800 & - & -
			& \phantom{a} & 0.1232 & 0.3707 & - & -
			& \phantom{a} & 0.1797 & 0.1880 & - & - \\
			TF-IDF & \phantom{a} & 0.1246 & 0.3109 & $5.4e^{-03**}$ & $-18.30$
			& \phantom{a} & 0.0581 & 0.2304 & $5.4e^{-10**}$ & $-52.84$
			& \phantom{a} & 0.0655 & 0.1040 & $6.9e^{-07**}$ & $-174.35$ \\
			PL2 & \phantom{a} & 0.1331 & 0.3218 & $6.0e^{-03**}$ & $-12.72$
			& \phantom{a} & 0.1289 & 0.3946 & $4.1e^{-01}$ & -
			& \phantom{a} & 0.1614 & 0.2000 & $3.3e^{-01*}$ & $-11.34$ \\
			PL2F & \phantom{a} & 0.1514 & 0.3473 & $7.7e^{-01}$ & -
			& \phantom{a} & 0.1023 & 0.3163 & $2.4e^{-04**}$ & $-16.96$
			& \phantom{a} & 0.1264 & 0.1560 & $4.5e^{-05**}$ & $-42.17$ \\
			\bottomrule
		\end{tabular}
	}
	\caption[Comparison of state-of-the-art functions against the MF generalizations]{Comparison of state-of-the-art candidates against the \gls{MF} generalizations. Mean Average Precision (MAP) and the Precision at 10 (P@10) scores of \gls{PL2MF} and \gls{BM25MF} and the other state-of-the-art candidates; a $p$-Value is computed using the two-tailed Wilcoxon matched-pairs signed-ranks test, where one star $*$ marks statistically significant difference at level $0.10$, and two stars $**$ at level $0.05$, with BM25MF (resp., PL2MF) used as a baseline; $\Delta\%$ indicates the difference in percentage between the two MAP values compared in that test.}
	\label{tab:ranking-cmp-results}
\end{table}



Table~\ref{tab:mf-effect} reports the MAP scores of BM25MF and PL2MF combined with each weight individually and using the normalization parameters values from the Table~\ref{tab:norm-param}. Apart from the row ``Without Attribute Label'', all runs consider the attribute label as an additional value as in the previous experiments.

\begin{table*}
\centering
\ra{0.5}
\resizebox{\textwidth}{!}{%
\begin{tabular}{cc@{\hs}rrccc@{\hs}rrccc@{\hs}rrcc}
\toprule
Method & \phantom{a} & \multicolumn{4}{c}{INEX09}
                     & \phantom{a} & \multicolumn{4}{c}{SS10}
                     & \phantom{a} & \multicolumn{4}{c}{SS11} \\
 & \phantom{a} & MAP & P@10 & $p$-Value & $\Delta\%$
               & \phantom{a} & MAP & P@10 & $p$-Value & $\Delta\%$
               & \phantom{a} & MAP & P@10 & $p$-Value & $\Delta\%$ \\
\cmidrule{3-6} \cmidrule{8-11} \cmidrule{13-16}
\emph{\textbf{BM25MF}} & \multicolumn{15}{c}{\phantom{a}} \\
{\raggedright With Attribute Label} & \phantom{a} & 0.1593 & 0.3982 & - & -
                                    & \phantom{a} & 0.1303 & 0.3783 & - & -
                                    & \phantom{a} & 0.1811 & 0.1880 & - & - \\
{\raggedright Without Attribute Label} & \phantom{a} & 0.1484 & 0.3800 & $6.9e^{-04**}$ & $-6.84$
                                       & \phantom{a} & 0.1241 & 0.3783 & $4.5e-01$ & -
                                       & \phantom{a} & 0.1763 & 0.1940 & $8.6e-01$ & - \\
\\
\emph{\textbf{BM25MF + QC}} & \multicolumn{15}{c}{\phantom{a}} \\
{\raggedright Value} & \phantom{a} & 0.1482 & 0.3545 & $2.0e^{-01*}$ & $-6.97$
                     & \phantom{a} & 0.1325 & 0.3793 & $8.2e^{-01}$ & -
                     & \phantom{a} & 0.1841 & 0.2020 & $1.6e^{-01*}$ & $+1.66$ \\
% & Value - No IDF & \phantom{a} & 0.1449 & 22 & 32 & 1 %&  %&
%                  & \phantom{a} & 0.1330 & 30 & 26 & 36 %& 1.3108 %& 0.2667
%                  & \phantom{a} & 0.1835 & 19 & 17 & 14 %& 1.7108 %& 0.3473
%                  \\
{\raggedright Attribute} & \phantom{a} & 0.1624 & 0.3818 & $8.8e^{-01}$ & -
                         & \phantom{a} & 0.1339 & 0.3815 & $3.5e^{-01*}$ & $+2.76$
                         & \phantom{a} & 0.1841 & 0.2060 & $5.3e^{-02*}$ & $+1.66$ \\
% & Attribute - No IDF & \phantom{a} & 0.1589 & 22 & 32 & 1 %&  %&
%                      & \phantom{a} & 0.1362 & 30 & 20 & 42 %& 1.9128 %& 0.2674
%                      & \phantom{a} & 0.1840 & 22 & 13 & 15 %& 1.9298 %& 0.3458
%                      \\
{\raggedright Entity} & \phantom{a} & 0.1514 & 0.3782 & $4.0e^{-02*}$ & $-4.96$
                      & \phantom{a} & 0.1236 & 0.3728 & $2.8e^{-02**}$ & $-5.14$
                      & \phantom{a} & 0.1744 & 0.188 & $4.7e^{-01}$ & - \\
% & Entity - No IDF & \phantom{a} & \underline{0.1449} & 16 & 38 & 1 %&  % &
%                   & \phantom{a} & 0.1265 & 4 & 11 & 77 %& 1.8354 % & 0.2448
%                   & \phantom{a} & 0.1743 & 10 & 18 & 22 %& 1.8584 % & 0.3386
%                   \\
{\raggedright All} & \phantom{a} & 0.1506 & 0.3545 & $1.5e^{-01*}$ & $-5.46$
                   & \phantom{a} & 0.1263 & 0.3717 & $3.1e^{-01*}$ & $-3.07$
                   & \phantom{a} & 0.1810 & 0.2080 & $7.1e^{-01}$ & - \\
% & All - No IDF & \phantom{a} & \underline{0.1392} & 21 & 34 & 0 %&  % &
%                & \phantom{a} & 0.1263 & 30 & 35 & 27 %& 1.7154 % & 0.2505
%                & \phantom{a} & 0.1744 & 15 & 24 & 11 %& 1.7458 % & 0.3304
%                \\
\\
\emph{\textbf{BM25MF + LC}} & \multicolumn{15}{c}{\phantom{a}} \\
\multirow{2}{*}{{\raggedright With Control Function (\ref{chap6:ranking:eq:lc-norm})}} & \phantom{a} & 0.1606 & 0.3964 & $2.8e^{-01*}$ & $+0.82$
                                                                 & \phantom{a} & 0.1321 & 0.3761 & $3.7e^{-01}$ & -
                                                                 & \phantom{a} & 0.1802 & 0.1920 & $5.4e^{-01}$ & - \\
 & \phantom{a} & \multicolumn{4}{c}{$n=1,\;\alpha=0.7$}
   & \phantom{a} & \multicolumn{4}{c}{$n=2,\;\alpha=0.4$}
   & \phantom{a} & \multicolumn{4}{c}{$n=1,\;\alpha=0.9$} \\
{\raggedright Without Control Function (\ref{chap6:ranking:eq:lc-norm})} & \phantom{a} & 0.1293 & 0.3055 & $8.9e^{-04**}$ & $-18.83$
                                                   & \phantom{a} & 0.1260 & 0.3609 & $5.3e^{-01}$ & -
                                                   & \phantom{a} & 0.1296 & 0.1820 & $2.1e^{-03**}$ & $-39.74$ \\
\\
\emph{\textbf{BM25MF + AEL}} & \multicolumn{15}{c}{\phantom{a}} \\
{\raggedright Entity Label Weight} & \phantom{a} & 0.1574 & 0.3909 & $9.1e^{-02*}$ & $-1.19$
                                   & \phantom{a} & 0.1401 & 0.4011 & $7.8e^{-05**}$ & $+7.52$
                                   & \phantom{a} & 0.1937 & 0.2000 & $6.1e^{-03**}$ & $+6.96$ \\
{\raggedright Attribute Label Weight} & \phantom{a} & 0.1604 & 0.3982 & $4.3e^{-01}$ & -
                                      & \phantom{a} & 0.1504 & 0.4228 & $2.6e^{-06**}$ & $+15.43$
                                      & \phantom{a} & 0.2173 & 0.2360 & $6.8e^{-06**}$ & $+19.99$ \\
{\raggedright Both} & \phantom{a} & 0.1593 & 0.3982 & $4.4e^{-01}$ & -
                    & \phantom{a} & 0.1584 & 0.4391 & $2.0e^{-07**}$ & $+21.57$
                    & \phantom{a} & 0.2274 & 0.2420 & $2.7e^{-05**}$ & $+25.57$ \\
\\
\emph{\textbf{BM25MF + QC + LC + AEL}} & \phantom{a} & 0.1589 &  & $5.9e^{-01}$ & -
                                       & \phantom{a} & 0.1720 & 0.4620 & $3.8e^{-06**}$ & $+32.00$
                                       & \phantom{a} & 0.2416 & 0.2560 & $1.1e^{-05**}$ & $+33.41$ \\
\midrule
\\[-0.2cm]

\emph{\textbf{PL2MF}} & \multicolumn{15}{c}{\phantom{a}} \\
{\raggedright With Attribute Label} & \phantom{a} & 0.1525 & 0.3800 & - & -
                                    & \phantom{a} & 0.1232 & 0.3685 & - & -
                                    & \phantom{a} & 0.1797 & 0.1880 & - & -
                        \\
{\raggedright Without Attribute Label} & \phantom{a} & 0.1401 & 0.3618 & $2.3e^{-04**}$ & $-8.13$
                                       & \phantom{a} & 0.1192 & 0.3674 & $5.3e^{-01}$ & -
                                       & \phantom{a} & 0.1680 & 0.1840 & $1.6e^{-01*}$ & $-6.51$
                           \\
\\
\emph{\textbf{PL2MF + QC}} & \multicolumn{15}{c}{\phantom{a}} \\
{\raggedright Value} & \phantom{a} & 0.1348 & 0.3382 & $1.0e^{-03**}$ & $-11.61$
                     & \phantom{a} & 0.1276 & 0.3750 & $2.8e^{-01*}$ & $+3.57$
                     & \phantom{a} & 0.1818 & 0.1980 & $1.6e^{-01*}$ & $+1.17$
         \\
% & Value - No IDF & \phantom{a} & \underline{0.1339} & 19 & 33 & 3 %&  %& 0.2892
%                  & \phantom{a} & \underline{0.1270} & 38 & 20 & 34 %& \underline{1.9616} %& 0.2530
%                  & \phantom{a} & \underline{0.1804} & 24 & 11 & 15 %& \underline{1.9606} %& 0.3477
%                  \\
{\raggedright Attribute} & \phantom{a} & 0.1569 & 0.3793 & $8.4e^{-01}$ & -
                         & \phantom{a} & 0.1299 & 0.3761 & $1.6e^{-01*}$ & $+5.44$
                         & \phantom{a} & 0.1815 & 0.1960 & $6.5e^{-02*}$ & $+1.00$
             \\
% & Attribute - No IDF & \phantom{a} & 0.1550 & 28 & 25 & 2 %& 0.9834 %& 0.3050
%                      & \phantom{a} & \underline{0.1270} & 32 & 17 & 43 %& \underline{1.9736} %& 0.2539
%                      & \phantom{a} & \underline{0.1796} & 23 & 12 & 15 %& \underline{1.9700} %& 0.3453
%                      \\
{\raggedright Entity} & \phantom{a} & 0.1499 & 0.3727 & $3.9e^{-01}$ & -
                      & \phantom{a} & 0.1168 & 0.3609 & $3.8e^{-02**}$ & $-5.19$
                      & \phantom{a} & 0.1655 & 0.1900 & $1.2e^{-01*}$ & $-7.90$
          \\
% & Entity - No IDF & \phantom{a} & 0.1431 & 24 & 30 & 1 %& 0.0892 %& 0.2884
%                   & \phantom{a} & 0.1175 & 5 & 9 & 78 %& 0.2584 %& 0.2400
%                   & \phantom{a} & 0.1630 & 8 & 20 & 22 %& 0.0174 %& 0.3278
%                   \\
{\raggedright All} & \phantom{a} & 0.1374 & 0.3309 & $1.5e^{-02**}$ & $-9.90$
                   & \phantom{a} & 0.1257 & 0.3728 & $6.4e^{-01}$ & -
                   & \phantom{a} & 0.1743 & 0.2020 & $9.4e^{-01}$ & $-3.01$
       \\
% & All - No IDF & \phantom{a} & \underline{0.1296} & 17 & 38 & 0 %& 0.0032 %& 0.2790
%                & \phantom{a} & 0.1211 & 32 & 31 & 29 %& 0.5754 %& 0.2426
%                & \phantom{a} & 0.1680 & 17 & 21 & 12 %& 0.4592 %& 0.3270
%                \\
\\
\emph{\textbf{PL2MF + LC}} & \multicolumn{15}{c}{\phantom{a}} \\
\multirow{2}{*}{{\raggedright With Function (\ref{chap6:ranking:eq:lc-norm})}} & \phantom{a} & 0.1494 & 0.3673 & $1.5e^{-02**}$ & $-2.03$
                                                & \phantom{a} & 0.1253 & 0.3663 & $2.4e^{-01*}$ & $+1.70$
                                                & \phantom{a} & 0.1802 & 0.1900 & $1.4e^{-01*}$ & $+0.29$ \\%[-0.05in]
 & \phantom{a} & \multicolumn{4}{c}{$n=1,\;\alpha=0.7$}
   & \phantom{a} & \multicolumn{4}{c}{$n=2,\;\alpha=0.4$}
   & \phantom{a} & \multicolumn{4}{c}{$n=1,\;\alpha=0.9$} \\
{\raggedright Without Function (\ref{chap6:ranking:eq:lc-norm})} & \phantom{a} & 0.1182 & 0.2909 & $5.4e^{-07**}$ & $-21.84$
                                                   & \phantom{a} & 0.1179 & 0.3391 & $3.8e^{-01}$ & -
                                                   & \phantom{a} & 0.1466 & 0.1900 & $7.3e^{-03**}$ & $-18.42$
                                       \\
\\
\emph{\textbf{PL2MF + AEL}} & \multicolumn{15}{c}{\phantom{a}} \\
{\raggedright Entity Label Weight} & \phantom{a} & 0.1523 & 0.3800 & $2.5e^{-01*}$ & $-0.13$
                                   & \phantom{a} & 0.1305 & 0.3848 & $6.4e^{-06**}$ & $+5.93$
                                   & \phantom{a} & 0.1824 & 0.1920 & $4.4e^{-02*}$ & $+1.50$
                       \\
{\raggedright Attribute Label Weight} & \phantom{a} & 0.1521 & 0.3673 & $1.9e^{-02**}$ & $-0.26$
                                      & \phantom{a} & 0.1471 & 0.4163 & $3.7e^{-07**}$ & $+19.40$
                                      & \phantom{a} & 0.2150 & 0.2320 & $4.0e^{-05**}$ & $+19.64$
                          \\
{\raggedright Both} & \phantom{a} & 0.1516 & 0.3709 & $1.2e^{-02**}$ & $-0.59$
                    & \phantom{a} & 0.1542 & 0.4337 & $3.4e^{-09**}$ & $+25.16$
                    & \phantom{a} & 0.2187 & 0.2380 & $1.2e^{-05**}$ & $+21.70$
        \\
\\
\emph{\textbf{PL2MF + QC + LC + AEL}} & \phantom{a} & 0.1492 & 0.3582 & $1.5e^{-01*}$ & $-2.16$
                                      & \phantom{a} & 0.1717 & 0.4620 & $2.0e^{-08**}$ & $+39.36$
                                      & \phantom{a} & 0.2360 & 0.2520 & $2.5e^{-05**}$ & $+31.33$ \\
\bottomrule
\end{tabular}}
\caption{Evaluation of the weights effectiveness on PL2MF and BM25MF.% effectiveness. Mean Average Precision (MAP) scores of each BM25MF and PL2MF with each weight individually; using the two-tailed Wilcoxon matched-pairs signed-ranks test, a $p$-Value with one star $*$ marks statistically significant difference at level $0.10$, and two stars $**$ at level $0.05$, with the baseline, i.e., BM25MF (respectively, PL2MF); $\Delta\%$ indicates the difference in percentage between the two MAP values.
}
\label{tab:mf-effect}
\end{table*}


\section{Graph Summary Ranking}
\label{app:summary-ranking}

The queries are grouped by its complexity, as outlined in Section~\ref{sec:summary-ranking:dataset}. A group is identified with the a string having the following regular expression: "\verb/\d+(-\d+)*/". For instance, the string ``\emph{2-1}'' identifies a group of queries that have \textit{two} star-shaped patterns, one with \textit{two} triple patterns and the other with only \textit{one}.

The SPARQL queries below are extracted from the logs available in the USEWOD2013 dataset \cite{usewod:2013}. We make use of the following prefixes:
\begin{labeling}{\textbf{yago:}}
	\item[\textbf{dbo:}] \url{http://dbpedia.org/ontology/}
	\item[\textbf{dbp:}] \url{http://dbpedia.org/property/}
	\item[\textbf{foaf:}] \url{http://xmlns.com/foaf/0.1/}
	\item[\textbf{geo:}] \url{http://www.w3.org/2003/01/geo/wgs84\_pos#}
	\item[\textbf{rdfs:}] \url{http://www.w3.org/2000/01/rdf-schema#}
	\item[\textbf{skos:}] \url{http://www.w3.org/2004/02/skos/core#}
	\item[\textbf{yago:}] \url{http://dbpedia.org/class/yago/}
\end{labeling}

Following are the SPARQL queries grouped by complexity:

\begin{table}
	\centering
	\rowcolors{2}{gray!25}{gray!8}
	\resizebox{\textwidth}{!}{
		\begin{tabular}{!{\color{white}\vrule}l!{\color{white}\vrule}c@{\hs}!{\color{white}\vrule}p{11.5cm}!{\color{white}\vrule}p{11.5cm}!{\color{white}\vrule}}
			\toprule
			Category & \phantom{a} & \multicolumn{2}{c}{Query} \\
			\cmidrule{1-1} \cmidrule{3-4}

			\rowcolor{white}
			1-1-2 & \phantom{a} & \\
			\cellcolor{white} & \cellcolor{white} \phantom{a} & \multicolumn{2}{p{23.45cm}}{?v0 skos:broader ?v1 .  ?v2 skos:broader ?v0 .  ?v3 skos:subject ?v2 .  ?v3 foaf:page ?v4 .} \\

			\midrule
			\rowcolor{white}
			1-1-3 & \phantom{a} & \\
			\cellcolor{white} & \cellcolor{white} \phantom{a} & \multicolumn{2}{p{23.45cm}}{?v0 skos:broader ?v1 .  ?v2 skos:broader ?v0 .  ?v3 skos:subject ?v2 .  ?v3 dbp:fuelType ?v4 .  ?v3 dbp:maxCapacity ?v5 .} \\
			\cellcolor{white} & \cellcolor{white} \phantom{a} & \multicolumn{2}{p{23.45cm}}{?v0 skos:broader ?v1 .  ?v2 skos:broader ?v0 .  ?v3 skos:subject ?v2 .  ?v3 foaf:page ?v4 .  ?v3 dbp:wikiPageUsesTemplate ?v5 .} \\

			\midrule
			\rowcolor{white}
			1-1-4 & \phantom{a} & \\
			\cellcolor{white} & \cellcolor{white} \phantom{a} & \multicolumn{2}{p{23.45cm}}{?v0 rdfs:label ?v1 .  ?v0 dbp:abstract ?v2 .  ?v0 dbp:city ?v3 .  ?v3 rdfs:label ?v4 .  ?v0 dbp:country ?v5 .  ?v5 rdfs:label ?v6 .} \\
			\cellcolor{white} & \cellcolor{white} \phantom{a} & \multicolumn{2}{p{23.45cm}}{?v0 skos:broader ?v1 .  ?v2 skos:broader ?v0 .  ?v3 skos:subject ?v2 .  ?v3 dbp:fuelType ?v4 .  ?v3 dbp:maxCapacity ?v5 .  ?v3 foaf:page ?v6 .} \\

			\midrule
			\rowcolor{white}
			1-1 & \phantom{a} & \\
			\cellcolor{white} & \cellcolor{white} \phantom{a} &  \cellcolor{gray!8} ?v0 dbo:artist ?v1 .  ?v1 dbp:abstract ?v2 . & \cellcolor{gray!25} ?v0 dbo:birthplace ?v1 .  ?v1 dbp:name ?v2 . \\
			\cellcolor{white} & \cellcolor{white} \phantom{a} &  \cellcolor{gray!25} ?v0 dbo:capital ?v1 .  ?v1 rdfs:label ?v2 . & \cellcolor{gray!8} ?v0 dbo:city ?v1 .  ?v1 rdfs:label ?v2 . \\
			\cellcolor{white} & \cellcolor{white} \phantom{a} &  \cellcolor{gray!8} ?v0 dbo:total\_type ?v1 .  ?v2 dbo:capital ?v0 . & \cellcolor{gray!25} ?v0 dbp:author ?v2 .  ?v2 rdfs:label ?v3 . \\
			\cellcolor{white} & \cellcolor{white} \phantom{a} &  \cellcolor{gray!25} ?v0 dbp:birthPlace ?v1 .  ?v1 rdfs:label ?v2 . & \cellcolor{gray!8} ?v0 dbp:capital ?v1 .  ?v1 rdfs:label ?v2 . \\
			\cellcolor{white} & \cellcolor{white} \phantom{a} &  \cellcolor{gray!8} ?v0 dbp:city ?v1 .  ?v1 rdfs:label ?v2 . & \cellcolor{gray!25} ?v0 dbp:incumbent ?v1 .  ?v1 rdfs:label ?v2 . \\
			\cellcolor{white} & \cellcolor{white} \phantom{a} &  \cellcolor{gray!25} ?v0 dbp:leaderName ?v1 .  ?v1 a dbo:Chancellor . & \cellcolor{gray!8} ?v0 dbp:party ?v1 .  ?v2 dbo:residence ?v1 . \\
			\cellcolor{white} & \cellcolor{white} \phantom{a} &  \cellcolor{gray!8} ?v0 dbp:redirect ?v1 .  ?v1 skos:subject ?v2 . & \cellcolor{gray!25} ?v0 dbp:starring ?v1 .  ?v2 dbp:director ?v1 . \\
			\cellcolor{white} & \cellcolor{white} \phantom{a} &  \cellcolor{gray!25} ?v0 foaf:name ?v1 .  ?v2 dbo:writer ?v0 . & \cellcolor{gray!8} ?v0 foaf:name ?v1 .  ?v2 dbp:notableRole ?v0 . \\
			\cellcolor{white} & \cellcolor{white} \phantom{a} &  \cellcolor{gray!8} ?v0 foaf:name ?v1 .  ?v2 dbp:portrayed ?v0 . & \cellcolor{gray!25} ?v0 foaf:name ?v1 .  ?v2 dbp:releaseDate ?v0 . \\
			\cellcolor{white} & \cellcolor{white} \phantom{a} &  \cellcolor{gray!25} ?v0 rdfs:label ?v1 .  ?v2 dbp:namedFor ?v0 . & \cellcolor{gray!8} ?v0 skos:broader ?v1 .  ?v1 rdfs:label ?v2 . \\
			\cellcolor{white} & \cellcolor{white} \phantom{a} &  \cellcolor{gray!8} ?v0 skos:broader ?v1 .  ?v2 skos:subject ?v0 . & \cellcolor{gray!25} ?v0 skos:subject ?v1 .  ?v1 rdfs:label ?v2 . \\

			\midrule
			\rowcolor{white}
			1-2 & \phantom{a} & \\
			\cellcolor{white} & \cellcolor{white} \phantom{a} & \cellcolor{gray!8}  ?v0 a dbo:Organisation .  ?v1 dbo:developer ?v0 .  ?v1 a dbo:Software . & \cellcolor{gray!25} ?v0 dbo:artist ?v1 .  ?v1 dbp:abstract ?v2 .  ?v1 foaf:page ?v3 . \\
			\cellcolor{white} & \cellcolor{white} \phantom{a} & \cellcolor{gray!25} ?v0 dbo:budget ?v1 .  ?v0 dbo:starring ?v2 .  ?v2 dbo:birthplace ?v3 . & \cellcolor{gray!8} ?v0 dbo:occupation ?v1 .  ?v2 a foaf:Person .  ?v2 dbp:workInstitution ?v1 . \\
			\cellcolor{white} & \cellcolor{white} \phantom{a} & \cellcolor{gray!8}  ?v0 dbo:total\_type ?v1 .  ?v2 dbo:capital ?v0 .  ?v2 rdfs:label ?v3 . & \cellcolor{gray!25} ?v0 dbp:book ?v3 .  ?v0 rdfs:label ?v2 .  ?v3 rdfs:label ?v4 . \\
			\cellcolor{white} & \cellcolor{white} \phantom{a} & \cellcolor{gray!25} ?v0 dbp:currency ?v3 .  ?v0 rdfs:label ?v2 .  ?v3 rdfs:label ?v4 . & \cellcolor{gray!8} ?v0 dbp:party ?v1 .  ?v2 a foaf:Person .  ?v2 dbo:residence ?v1 . \\
			\cellcolor{white} & \cellcolor{white} \phantom{a} & \cellcolor{gray!8}  ?v0 dbp:president ?v3 .  ?v0 rdfs:label ?v2 .  ?v3 rdfs:label ?v4 . & \cellcolor{gray!25} ?v0 dbp:primeMinister ?v3 .  ?v0 rdfs:label ?v2 .  ?v3 rdfs:label ?v4 . \\
			\cellcolor{white} & \cellcolor{white} \phantom{a} & \cellcolor{gray!25} ?v0 dbp:state ?v3 .  ?v0 rdfs:label ?v2 .  ?v3 rdfs:label ?v4 . & \cellcolor{gray!8} ?v0 dbp:type ?v3 .  ?v0 rdfs:label ?v2 .  ?v3 rdfs:label ?v4 . \\
			\cellcolor{white} & \cellcolor{white} \phantom{a} & \cellcolor{gray!8}  ?v0 rdfs:label ?v1 .  ?v0 dbp:author ?v3 .  ?v3 rdfs:label ?v4 . & \cellcolor{gray!25} ?v0 rdfs:label ?v1 .  ?v0 dbp:capital ?v3 .  ?v3 rdfs:label ?v4 . \\
			\cellcolor{white} & \cellcolor{white} \phantom{a} & \cellcolor{gray!25} ?v0 rdfs:label ?v1 .  ?v0 dbp:name ?v3 .  ?v3 rdfs:label ?v4 . & \cellcolor{gray!8} ?v0 rdfs:label ?v1 .  ?v0 skos:subject ?v2 .  ?v2 skos:broader ?v3 . \\
			\cellcolor{white} & \cellcolor{white} \phantom{a} & \cellcolor{gray!8}  ?v0 rdfs:label ?v1 .  ?v2 skos:subject ?v0 .  ?v2 rdfs:label ?v3 . & \cellcolor{gray!25} ?v0 skos:broader ?v1 .  ?v2 skos:subject ?v0 .  ?v2 foaf:img ?v3 . \\
			\cellcolor{white} & \cellcolor{white} \phantom{a} & \cellcolor{gray!25} ?v0 skos:subject ?v1 .  ?v1 rdfs:label ?v2 .  ?v0 dbp:blackboard ?v3 . & \\

			\midrule
			\rowcolor{white}
			1-3 & \phantom{a} & \\
			\cellcolor{white} & \cellcolor{white} \phantom{a} & \multicolumn{2}{p{23.45cm}}{?v0 skos:broader ?v1 .  ?v2 skos:subject ?v0 .  ?v2 dbp:fuelType ?v3 .  ?v2 dbp:maxCapacity ?v4 .} \\

			\midrule
			\rowcolor{white}
			1-4 & \phantom{a} & \\
			\cellcolor{white} & \cellcolor{white} \phantom{a} & \multicolumn{2}{p{23.45cm}}{?v0 a dbo:Film .  ?v0 dbo:director ?v1 .  ?v0 foaf:name ?v2 .  ?v0 dbp:imdbId ?v3 .  ?v1 dbp:dateOfBirth ?v4 .} \\

			\midrule
			\rowcolor{white}
			1-5 & \phantom{a} & \\
			\cellcolor{white} & \cellcolor{white} \phantom{a} & \multicolumn{2}{p{23.45cm}}{?v0 dbp:artist ?v1 .  ?v0 dbp:name ?v2 .  ?v0 foaf:page ?v3 .  ?v0 dbp:released ?v4 .  ?v0 dbp:cover ?v5 .  ?v1 rdfs:label ?v6 .} \\

			\midrule
			\rowcolor{white}
			2-2 & \phantom{a} & \\
			\cellcolor{white} & \cellcolor{white} \phantom{a} & \multicolumn{2}{p{23.45cm}}{?v0 skos:subject ?v1 .  ?v0 dbp:capital ?v2 .  ?v2 geo:lat ?v3 .  ?v2 geo:long ?v4 .} \\
			\cellcolor{white} & \cellcolor{white} \phantom{a} & \multicolumn{2}{p{23.45cm}}{?v0 skos:subject ?v1 .  ?v0 dbp:highestpoint ?v2 .  ?v2 geo:lat ?v3 .  ?v2 geo:long ?v4 .} \\

			\midrule
			\rowcolor{white}
			3-4 & \phantom{a} & \\
			\cellcolor{white} & \cellcolor{white} \phantom{a} & \multicolumn{2}{p{23.45cm}}{?v0 dbp:name ?v1 .  ?v0 skos:subject ?v2 .  ?v0 dbp:genre ?v3 .  ?v0 dbp:origin ?v4 .  ?v5 dbp:city ?v4 .  ?v5 skos:subject ?v6 .  ?v5 dbp:address ?v7 .} \\
			\bottomrule
		\end{tabular}
	}
	\caption{Queries of complexity 1-1-2, 1-1-3, 1-1-4, 1-1, 1-2, 1-3, 1-4, 1-5, 2-2, 3-4}
	\label{appendix:ranking:tab:queries1}
\end{table}


\begin{table}
	\centering
	\rowcolors{2}{gray!25}{gray!8}
	\resizebox{\textwidth}{!}{
		\begin{tabular}{!{\color{white}\vrule}l!{\color{white}\vrule}c@{\hs}!{\color{white}\vrule}p{11.5cm}!{\color{white}\vrule}p{11.5cm}!{\color{white}\vrule}}
			\toprule
			Category & \phantom{a} & \multicolumn{2}{c}{Query} \\
			\cmidrule{1-1} \cmidrule{3-4}
			
			\rowcolor{white}
			2 & \phantom{a} & \\
			\cellcolor{white} & \cellcolor{white} \phantom{a} & \cellcolor{gray!8}    ?v0 a dbo:Actor .  ?v0 rdfs:label ?v1 . & \cellcolor{gray!25} ?v0 a dbo:Beverage  .  ?v0 rdfs:label ?v1 . \\
			\cellcolor{white} & \cellcolor{white} \phantom{a} & \cellcolor{gray!25}    ?v0 a dbo:ChemicalCompound .  ?v0 rdfs:label ?v1 . & \cellcolor{gray!8} ?v0 a dbo:Currency .  ?v0 rdfs:label ?v1 . \\
			\cellcolor{white} & \cellcolor{white} \phantom{a} & \cellcolor{gray!8}    ?v0 a dbo:Disease  .  ?v0 rdfs:label ?v1 . & \cellcolor{gray!25} ?v0 a dbo:Film .  ?v0 rdfs:label ?v1 . \\
			\cellcolor{white} & \cellcolor{white} \phantom{a} & \cellcolor{gray!25}    ?v0 a dbo:MusicalArtist .  ?v0 dbp:abstract ?v1 . & \cellcolor{gray!8} ?v0 a dbo:Person  .  ?v0 dbo:weight ?v1 . \\
			\cellcolor{white} & \cellcolor{white} \phantom{a} & \cellcolor{gray!8}    ?v0 a dbo:PrimeMinister .  ?v0 dbo:deathdate ?v1 . & \cellcolor{gray!25} ?v0 a foaf:Person .  ?v0 dbo:residence ?v1 . \\
			\cellcolor{white} & \cellcolor{white} \phantom{a} & \cellcolor{gray!25}    ?v0 a foaf:Person .  ?v0 dbp:party ?v1 . & \cellcolor{gray!8} ?v0 a foaf:Person .  ?v0 dbp:residence ?v1 . \\
			\cellcolor{white} & \cellcolor{white} \phantom{a} & \cellcolor{gray!8}    ?v0 a yago:ChancellorsOfGermany .  ?v0 dbp:occupation ?v1 . & \cellcolor{gray!25} ?v0 a yago:Company108058098 .  ?v0 foaf:homepage ?v1 . \\
			\cellcolor{white} & \cellcolor{white} \phantom{a} & \cellcolor{gray!25}    ?v0 a yago:Country108544813  .  ?v0 dbp:establishedDate ?v1 . & \cellcolor{gray!8} ?v0 a yago:HostCitiesOfTheSummerOlympicGames .  ?v0 rdfs:label ?v1 . \\
			\cellcolor{white} & \cellcolor{white} \phantom{a} & \cellcolor{gray!8}    ?v0 dbo:author ?v1 .  ?v0 foaf:page ?v2 . & \cellcolor{gray!25} ?v0 dbo:birthdate ?v1 .  ?v0 dbp:name ?v2 . \\
			\cellcolor{white} & \cellcolor{white} \phantom{a} & \cellcolor{gray!25}    ?v0 dbo:birthdate ?v1 .  ?v0 dbp:profession ?v2 . & \cellcolor{gray!8} ?v0 dbo:birthdate ?v1 .  ?v0 foaf:name ?v2 . \\
			\cellcolor{white} & \cellcolor{white} \phantom{a} & \cellcolor{gray!8}    ?v0 dbo:birthplace ?v1 .  ?v0 foaf:name ?v2 . & \cellcolor{gray!25} ?v0 dbo:birthplace ?v1 .  ?v0 rdfs:label ?v2 . \\
			\cellcolor{white} & \cellcolor{white} \phantom{a} & \cellcolor{gray!25}    ?v0 dbo:budget ?v1 .  ?v0 dbo:releaseDate ?v2 . & \cellcolor{gray!8} ?v0 dbo:deathdate ?v1 .  ?v0 rdfs:label ?v2 . \\
			\cellcolor{white} & \cellcolor{white} \phantom{a} & \cellcolor{gray!8}    ?v0 dbo:deathplace ?v1 .  ?v0 rdfs:label ?v2 . & \cellcolor{gray!25} ?v0 dbo:developer ?v1 .  ?v0 a dbo:Software . \\
			\cellcolor{white} & \cellcolor{white} \phantom{a} & \cellcolor{gray!25}    ?v0 dbo:director ?v1 .  ?v0 a dbo:Film . & \cellcolor{gray!8} ?v0 dbo:industry ?v1 .  ?v0 dbo:revenue ?v2 . \\
			\cellcolor{white} & \cellcolor{white} \phantom{a} & \cellcolor{gray!8}    ?v0 dbo:industry ?v1 .  ?v0 dbp:revenue ?v2 . & \cellcolor{gray!25} ?v0 dbo:industry ?v1 .  ?v0 rdfs:label ?v2 . \\
			\cellcolor{white} & \cellcolor{white} \phantom{a} & \cellcolor{gray!25}    ?v0 dbo:influences ?v1 .  ?v0 foaf:page ?v2 . & \cellcolor{gray!8} ?v0 dbo:numberOfEmployees ?v1 .  ?v0 dbo:revenue ?v2 . \\
			\cellcolor{white} & \cellcolor{white} \phantom{a} & \cellcolor{gray!8}    ?v0 dbo:occupation ?v1 .  ?v0 a foaf:Person . & \cellcolor{gray!25} ?v0 dbo:releaseDate ?v1 .  ?v0 dbo:episodeNumber ?v2 . \\
			\cellcolor{white} & \cellcolor{white} \phantom{a} & \cellcolor{gray!25}    ?v0 dbo:starring ?v1 .  ?v0 dbo:budget ?v2 . & \cellcolor{gray!8} ?v0 dbo:starring ?v1 .  ?v0 dbo:releaseDate ?v2 . \\
			\cellcolor{white} & \cellcolor{white} \phantom{a} & \cellcolor{gray!8}    ?v0 dbp:abstract ?v1 .  ?v0 dbp:flag ?v2 . & \cellcolor{gray!25} ?v0 dbp:abstract ?v1 .  ?v0 rdfs:label ?v2 . \\
			\cellcolor{white} & \cellcolor{white} \phantom{a} & \cellcolor{gray!25}    ?v0 dbp:airdate ?v1 .  ?v0 dbp:episodeName ?v2 . & \cellcolor{gray!8} ?v0 dbp:almaMater ?v1 .  ?v0 rdfs:label ?v2 . \\
			\cellcolor{white} & \cellcolor{white} \phantom{a} & \cellcolor{gray!8}    ?v0 dbp:associatedActs ?v1 .  ?v0 rdfs:label ?v2 . & \cellcolor{gray!25} ?v0 dbp:birth ?v1 .  ?v0 foaf:name ?v2 . \\
			\cellcolor{white} & \cellcolor{white} \phantom{a} & \cellcolor{gray!25}    ?v0 dbp:birthPlace ?v1 .  ?v0 dbo:birthdate ?v2 . & \cellcolor{gray!8} ?v0 dbp:birthPlace ?v1 .  ?v0 dbp:deathPlace ?v1 . \\
			\cellcolor{white} & \cellcolor{white} \phantom{a} & \cellcolor{gray!8}    ?v0 dbp:birthPlace ?v1 .  ?v0 dbp:name ?v2 . & \cellcolor{gray!25} ?v0 dbp:birthPlace ?v1 .  ?v0 foaf:name ?v2 . \\
			\cellcolor{white} & \cellcolor{white} \phantom{a} & \cellcolor{gray!25}    ?v0 dbp:book ?v1 .  ?v0 rdfs:label ?v2 . & \cellcolor{gray!8} ?v0 dbp:callingCode ?v1 .  ?v0 rdfs:label ?v2 . \\
			\cellcolor{white} & \cellcolor{white} \phantom{a} & \cellcolor{gray!8}    ?v0 dbp:city ?v1 .  ?v0 rdfs:label ?v2 . & \cellcolor{gray!25} ?v0 dbp:country ?v1 .  ?v0 rdfs:label ?v2 . \\
			\cellcolor{white} & \cellcolor{white} \phantom{a} & \cellcolor{gray!25}    ?v0 dbp:creator ?v1 .  ?v0 dbp:starring ?v2 . & \cellcolor{gray!8} ?v0 dbp:currency ?v1 .  ?v0 rdfs:label ?v2 . \\
			\cellcolor{white} & \cellcolor{white} \phantom{a} & \cellcolor{gray!8}    ?v0 dbp:currentMembers ?v1 .  ?v0 rdfs:label ?v2 . & \cellcolor{gray!25} ?v0 dbp:episodeNo ?v1 .  ?v0 dbp:airdate ?v2 . \\
			\cellcolor{white} & \cellcolor{white} \phantom{a} & \cellcolor{gray!25}    ?v0 dbp:fuelType ?v1 .  ?v0 dbp:maxCapacity ?v2 . & \cellcolor{gray!8} ?v0 dbp:genre ?v1 .  ?v0 foaf:name ?v2 . \\
			\cellcolor{white} & \cellcolor{white} \phantom{a} & \cellcolor{gray!8}    ?v0 dbp:genre ?v1 .  ?v0 rdfs:label ?v2 . & \cellcolor{gray!25} ?v0 dbp:gini ?v1 .  ?v0 a dbo:Country . \\
			\cellcolor{white} & \cellcolor{white} \phantom{a} & \cellcolor{gray!25}    ?v0 dbp:height ?v1 .  ?v0 dbo:birthplace ?v2 . & \cellcolor{gray!8} ?v0 dbp:height ?v1 .  ?v0 dbp:wikiPageUsesTemplate ?v2 . \\
			\cellcolor{white} & \cellcolor{white} \phantom{a} & \cellcolor{gray!8}    ?v0 dbp:homepage ?v1 .  ?v0 dbo:location ?v2 . & \cellcolor{gray!25} ?v0 dbp:imdbTitleProperty ?v1 .  ?v0 a dbo:Film . \\
			\cellcolor{white} & \cellcolor{white} \phantom{a} & \cellcolor{gray!25}    ?v0 dbp:label ?v1 .  ?v0 dbp:website ?v2 . & \cellcolor{gray!8} ?v0 dbp:label ?v1 .  ?v0 rdfs:label ?v2 . \\
			\cellcolor{white} & \cellcolor{white} \phantom{a} & \cellcolor{gray!8}    ?v0 dbp:lenght ?v1 .  ?v0 rdfs:label ?v2 . & \cellcolor{gray!25} ?v0 dbp:location ?v1 .  ?v0 skos:subject ?v2 . \\
			\cellcolor{white} & \cellcolor{white} \phantom{a} & \cellcolor{gray!25}    ?v0 dbp:mainInterests ?v1 .  ?v0 foaf:page ?v2 . & \cellcolor{gray!8} ?v0 dbp:name ?v1 .  ?v0 dbp:abstract ?v2 . \\
			\cellcolor{white} & \cellcolor{white} \phantom{a} & \cellcolor{gray!8}    ?v0 dbp:party ?v1 .  ?v0 dbo:residence ?v2 . & \cellcolor{gray!25} ?v0 dbp:pastMembers ?v1 .  ?v0 rdfs:label ?v2 . \\
			\cellcolor{white} & \cellcolor{white} \phantom{a} & \cellcolor{gray!25}    ?v0 dbp:primeMinister ?v1 .  ?v0 rdfs:label ?v2 . & \cellcolor{gray!8} ?v0 dbp:starring ?v1 .  ?v0 a dbo:Film . \\
			\cellcolor{white} & \cellcolor{white} \phantom{a} & \cellcolor{gray!8}    ?v0 dbp:starring ?v1 .  ?v0 dbp:director ?v1 . & \cellcolor{gray!25} ?v0 dbp:starring ?v1 .  ?v0 skos:subject ?v2 . \\
			\cellcolor{white} & \cellcolor{white} \phantom{a} & \cellcolor{gray!25}    ?v0 dbp:startDateProperty ?v1 .  ?v0 dbp:name ?v2 . & \cellcolor{gray!8} ?v0 dbp:states ?v1 .  ?v0 a dbo:Country . \\
			\cellcolor{white} & \cellcolor{white} \phantom{a} & \cellcolor{gray!8}    ?v0 dbp:wikiPageUsesTemplate ?v1 .  ?v0 dbp:name ?v2 . & \cellcolor{gray!25} ?v0 dbp:writer ?v1 .  ?v0 dbp:airdate ?v2 . \\
			\cellcolor{white} & \cellcolor{white} \phantom{a} & \cellcolor{gray!25}    ?v0 dbp:writer ?v1 .  ?v0 dbp:blackboard ?v2 . & \cellcolor{gray!8} ?v0 foaf:depiction ?v1 .  ?v0 rdfs:label ?v2 . \\
			\cellcolor{white} & \cellcolor{white} \phantom{a} & \cellcolor{gray!8}    ?v0 foaf:homepage ?v1 .  ?v0 a yago:University108286163 . & \cellcolor{gray!25} ?v0 foaf:homepage ?v1 .  ?v0 foaf:name ?v2 . \\
			\cellcolor{white} & \cellcolor{white} \phantom{a} & \cellcolor{gray!25}    ?v0 foaf:name ?v1 .  ?v0 a dbo:City . & \cellcolor{gray!8} ?v0 foaf:name ?v1 .  ?v0 a dbo:Person . \\
			\cellcolor{white} & \cellcolor{white} \phantom{a} & \cellcolor{gray!8}    ?v0 foaf:name ?v1 .  ?v0 a dbo:Place . & \cellcolor{gray!25} ?v0 foaf:name ?v1 .  ?v0 a foaf:Person . \\
			\cellcolor{white} & \cellcolor{white} \phantom{a} & \cellcolor{gray!25}    ?v0 foaf:name ?v1 .  ?v0 a yago:University108286163 . & \cellcolor{gray!8} ?v0 foaf:name ?v1 .  ?v0 dbo:deathdate ?v2 . \\
			\cellcolor{white} & \cellcolor{white} \phantom{a} & \cellcolor{gray!8}    ?v0 foaf:name ?v1 .  ?v0 dbo:episodenumber ?v2 . & \cellcolor{gray!25} ?v0 foaf:name ?v1 .  ?v0 dbo:guest ?v2 . \\
			\cellcolor{white} & \cellcolor{white} \phantom{a} & \cellcolor{gray!25}    ?v0 foaf:name ?v1 .  ?v0 dbo:industry ?v2 . & \cellcolor{gray!8} ?v0 foaf:name ?v1 .  ?v0 dbo:numberOfEmployees ?v2 . \\
			\cellcolor{white} & \cellcolor{white} \phantom{a} & \cellcolor{gray!8}    ?v0 foaf:name ?v1 .  ?v0 dbo:releaseDate ?v2 . & \cellcolor{gray!25} ?v0 foaf:name ?v1 .  ?v0 dbo:starring ?v2 . \\
			\cellcolor{white} & \cellcolor{white} \phantom{a} & \cellcolor{gray!25}    ?v0 foaf:name ?v1 .  ?v0 dbo:weight ?v2 . & \cellcolor{gray!8} ?v0 foaf:name ?v1 .  ?v0 dbo:writer ?v2 . \\
			\cellcolor{white} & \cellcolor{white} \phantom{a} & \cellcolor{gray!8}    ?v0 foaf:name ?v1 .  ?v0 dbp:abstract ?v2 . & \cellcolor{gray!25} ?v0 foaf:name ?v1 .  ?v0 dbp:creator ?v2 . \\
			\cellcolor{white} & \cellcolor{white} \phantom{a} & \cellcolor{gray!25}    ?v0 foaf:name ?v1 .  ?v0 dbp:episodenumber ?v2 . & \cellcolor{gray!8} ?v0 foaf:name ?v1 .  ?v0 dbp:firstAired ?v2 . \\
			\cellcolor{white} & \cellcolor{white} \phantom{a} & \cellcolor{gray!8}    ?v0 foaf:name ?v1 .  ?v0 dbp:notableRole ?v2 . & \cellcolor{gray!25} ?v0 foaf:name ?v1 .  ?v0 dbp:notableWorks ?v2 . \\
			\cellcolor{white} & \cellcolor{white} \phantom{a} & \cellcolor{gray!25}    ?v0 foaf:name ?v1 .  ?v0 dbp:origin ?v2 . & \cellcolor{gray!8} ?v0 foaf:name ?v1 .  ?v0 dbp:releaseDate ?v2 . \\
			\cellcolor{white} & \cellcolor{white} \phantom{a} & \cellcolor{gray!8}    ?v0 foaf:name ?v1 .  ?v0 dbp:starring ?v2 . & \cellcolor{gray!25} ?v0 foaf:name ?v1 .  ?v0 foaf:img ?v2 . \\
			\cellcolor{white} & \cellcolor{white} \phantom{a} & \cellcolor{gray!25}    ?v0 foaf:name ?v1 .  ?v0 foaf:page ?v2 . & \cellcolor{gray!8} ?v0 foaf:name ?v1 .  ?v0 foaf:surname ?v2 . \\
			\cellcolor{white} & \cellcolor{white} \phantom{a} & \cellcolor{gray!8}    ?v0 foaf:name ?v1 .  ?v0 rdfs:label ?v2 . & \cellcolor{gray!25} ?v0 foaf:name ?v1 .  ?v0 skos:subject ?v2 . \\
			\cellcolor{white} & \cellcolor{white} \phantom{a} & \cellcolor{gray!25}    ?v0 foaf:page ?v1 .  ?v0 dbp:abstract ?v2 . & \cellcolor{gray!8} ?v0 foaf:page ?v1 .  ?v0 rdfs:label ?v2 . \\
			\cellcolor{white} & \cellcolor{white} \phantom{a} & \cellcolor{gray!8}    ?v0 geo:lat ?v1 .  ?v0 geo:long ?v2 . & \cellcolor{gray!25} ?v0 geo:long ?v1 .  ?v0 rdfs:label ?v2 . \\
			\cellcolor{white} & \cellcolor{white} \phantom{a} & \cellcolor{gray!25}    ?v0 rdfs:comment ?v1 .  ?v0 a dbo:City . & \cellcolor{gray!8} ?v0 rdfs:label ?v1 .  ?v0 a dbo:Person . \\
			\cellcolor{white} & \cellcolor{white} \phantom{a} & \cellcolor{gray!8}    ?v0 rdfs:label ?v1 .  ?v0 dbp:animal ?v2 . & \cellcolor{gray!25} ?v0 rdfs:label ?v1 .  ?v0 dbp:areaKm ?v2 . \\
			\cellcolor{white} & \cellcolor{white} \phantom{a} & \cellcolor{gray!25}    ?v0 rdfs:label ?v1 .  ?v0 dbp:author ?v2 . & \cellcolor{gray!8} ?v0 rdfs:label ?v1 .  ?v0 dbp:award ?v2 . \\
			\cellcolor{white} & \cellcolor{white} \phantom{a} & \cellcolor{gray!8}    ?v0 rdfs:label ?v1 .  ?v0 dbp:birth ?v2 . & \cellcolor{gray!25} ?v0 rdfs:label ?v1 .  ?v0 dbp:presidentStart ?v2 . \\
			\cellcolor{white} & \cellcolor{white} \phantom{a} & \cellcolor{gray!25}    ?v0 rdfs:label ?v1 .  ?v0 dbp:reference ?v2 . & \cellcolor{gray!8} ?v0 rdfs:label ?v1 .  ?v0 rdfs:comment ?v2 . \\
			\cellcolor{white} & \cellcolor{white} \phantom{a} & \cellcolor{gray!8}    ?v0 rdfs:label ?v1 .  ?v0 skos:prefLabel ?v2 . & \cellcolor{gray!25} ?v0 skos:broader ?v1 .  ?v0 rdfs:label ?v2 . \\
			\cellcolor{white} & \cellcolor{white} \phantom{a} & \cellcolor{gray!25}    ?v0 skos:subject ?v1 .  ?v0 dbp:abstract ?v2 . & \cellcolor{gray!8} ?v0 skos:subject ?v1 .  ?v0 dbp:airdate ?v2 . \\
			\cellcolor{white} & \cellcolor{white} \phantom{a} & \cellcolor{gray!8}    ?v0 skos:subject ?v1 .  ?v0 dbp:blackboard ?v2 . & \cellcolor{gray!25} ?v0 skos:subject ?v1 .  ?v0 dbp:capital ?v2 . \\
			\cellcolor{white} & \cellcolor{white} \phantom{a} & \cellcolor{gray!25}    ?v0 skos:subject ?v1 .  ?v0 dbp:director ?v2 . & \cellcolor{gray!8} ?v0 skos:subject ?v1 .  ?v0 foaf:depiction ?v2 . \\
			\cellcolor{white} & \cellcolor{white} \phantom{a} & \cellcolor{gray!8}    ?v0 skos:subject ?v1 .  ?v0 foaf:page ?v2 . & \cellcolor{gray!25} ?v0 skos:subject ?v1 .  ?v0 rdfs:label ?v2 . \\
			\bottomrule
		\end{tabular}
	}
\end{table}

\begin{table}
	\centering
	\rowcolors{2}{gray!25}{gray!8}
	\resizebox{\textwidth}{!}{
		\begin{tabular}{!{\color{white}\vrule}l!{\color{white}\vrule}c@{\hs}!{\color{white}\vrule}p{11.5cm}!{\color{white}\vrule}p{11.5cm}!{\color{white}\vrule}}
			\toprule
			Category & \phantom{a} & \multicolumn{2}{c}{Query} \\
			\cmidrule{1-1} \cmidrule{3-4}
			
			\rowcolor{white}
			3 & \phantom{a} & \\
			\cellcolor{white} & \cellcolor{white} \phantom{a} & \cellcolor{gray!8}    ?v0 a dbo:Artist .  ?v0 rdfs:label ?v1 .  ?v0 dbp:abstract ?v2 . & \cellcolor{gray!25} ?v0 a dbo:Country .  ?v0 dbp:iso31661Alpha ?v1 .  ?v0 dbp:populationCensus ?v2 . \\
			\cellcolor{white} & \cellcolor{white} \phantom{a} & \cellcolor{gray!25}    ?v0 a dbo:Country .  ?v0 rdfs:label ?v1 .  ?v0 dbp:percentWater ?v2 . & \cellcolor{gray!8} ?v0 a dbo:MilitaryConflict .  ?v0 rdfs:label ?v1 .  ?v0 dbo:date ?v2 . \\
			\cellcolor{white} & \cellcolor{white} \phantom{a} & \cellcolor{gray!8}    ?v0 a dbo:MusicalArtist .  ?v0 dbp:abstract ?v1 .  ?v0 foaf:page ?v2 . & \cellcolor{gray!25} ?v0 a dbo:Person .  ?v0 dbo:birthdate ?v1 .  ?v0 foaf:name ?v2 . \\
			\cellcolor{white} & \cellcolor{white} \phantom{a} & \cellcolor{gray!25}    ?v0 a dbo:PrimeMinister .  ?v0 foaf:name ?v1 .  ?v0 dbo:deathdate ?v2 . & \cellcolor{gray!8} ?v0 a dbo:SkiArea .  ?v0 rdfs:label ?v1 .  ?v0 dbp:liftsystem ?v2 . \\
			\cellcolor{white} & \cellcolor{white} \phantom{a} & \cellcolor{gray!8}    ?v0 a dbo:Writer .  ?v0 rdfs:label ?v1 .  ?v0 dbp:abstract ?v2 . & \cellcolor{gray!25} ?v0 a yago:LandlockedCountries .  ?v0 rdfs:label ?v1 .  ?v0 dbp:populationEstimate ?v2 . \\
			\cellcolor{white} & \cellcolor{white} \phantom{a} & \cellcolor{gray!25}    ?v0 dbo:author ?v1 .  ?v0 foaf:page ?v2 .  ?v0 rdfs:label ?v3 . & \cellcolor{gray!8} ?v0 dbo:birthdate ?v1 .  ?v0 dbo:deathdate ?v2 .  ?v0 foaf:name ?v3 . \\
			\cellcolor{white} & \cellcolor{white} \phantom{a} & \cellcolor{gray!8}    ?v0 dbo:birthdate ?v1 .  ?v0 foaf:name ?v2 .  ?v0 dbo:movement ?v3 . & \cellcolor{gray!25} ?v0 dbo:birthdate ?v1 .  ?v0 foaf:name ?v2 .  ?v0 dbp:movement ?v3 . \\
			\cellcolor{white} & \cellcolor{white} \phantom{a} & \cellcolor{gray!25}    ?v0 dbo:birthdate ?v1 .  ?v0 foaf:name ?v2 .  ?v0 dbp:ocupation ?v3 . & \cellcolor{gray!8} ?v0 dbo:birthdate ?v1 .  ?v0 foaf:name ?v2 .  ?v0 rdfs:comment ?v3 . \\
			\cellcolor{white} & \cellcolor{white} \phantom{a} & \cellcolor{gray!8}    ?v0 dbo:birthdate ?v1 .  ?v0 foaf:name ?v2 .  ?v0 skos:subject ?v3 . & \cellcolor{gray!25} ?v0 dbo:birthplace ?v1 .  ?v0 dbo:birthdate ?v2 .  ?v0 dbp:profession ?v3 . \\
			\cellcolor{white} & \cellcolor{white} \phantom{a} & \cellcolor{gray!25}    ?v0 dbo:birthplace ?v1 .  ?v0 dbo:birthdate ?v2 .  ?v0 dbp:spouse ?v3 . & \cellcolor{gray!8} ?v0 dbo:birthplace ?v1 .  ?v0 dbp:abstract ?v2 .  ?v0 foaf:name ?v3 . \\
			\cellcolor{white} & \cellcolor{white} \phantom{a} & \cellcolor{gray!8}    ?v0 dbo:birthplace ?v1 .  ?v0 foaf:name ?v2 .  ?v0 dbo:deathdate ?v3 . & \cellcolor{gray!25} ?v0 dbo:birthplace ?v1 .  ?v0 foaf:name ?v2 .  ?v0 rdfs:comment ?v3 . \\
			\cellcolor{white} & \cellcolor{white} \phantom{a} & \cellcolor{gray!25}    ?v0 dbo:foundationplace ?v1 .  ?v0 a dbo:Organisation .  ?v0 rdfs:label ?v2 . & \cellcolor{gray!8} ?v0 dbo:imdbid ?v1 .  ?v0 a dbo:Film .  ?v0 dbo:releaseDate ?v2 . \\
			\cellcolor{white} & \cellcolor{white} \phantom{a} & \cellcolor{gray!8}    ?v0 dbo:influenced ?v1 .  ?v0 foaf:page ?v2 .  ?v0 rdfs:label ?v3 . & \cellcolor{gray!25} ?v0 dbo:influences ?v1 .  ?v0 foaf:page ?v2 .  ?v0 foaf:name ?v3 . \\
			\cellcolor{white} & \cellcolor{white} \phantom{a} & \cellcolor{gray!25}    ?v0 dbo:influences ?v1 .  ?v0 foaf:page ?v2 .  ?v0 rdfs:label ?v3 . & \cellcolor{gray!8} ?v0 dbo:numberOfEmployees ?v1 .  ?v0 dbo:revenue ?v2 .  ?v0 dbp:foundation ?v3 . \\
			\cellcolor{white} & \cellcolor{white} \phantom{a} & \cellcolor{gray!8}    ?v0 dbo:releaseDate ?v1 .  ?v0 dbo:episodeNumber ?v2 .  ?v0 dbo:seasonNumber ?v3 . & \cellcolor{gray!25} ?v0 dbo:series ?v1 .  ?v0 dbo:releaseDate ?v2 .  ?v0 dbo:episodeNumber ?v3 . \\
			\cellcolor{white} & \cellcolor{white} \phantom{a} & \cellcolor{gray!25}    ?v0 dbo:starring ?v1 .  ?v0 dbo:budget ?v2 .  ?v0 dbo:releaseDate ?v3 . & \cellcolor{gray!8} ?v0 dbo:writer ?v1 .  ?v0 dbo:director ?v2 .  ?v0 dbp:date ?v3 . \\
			\cellcolor{white} & \cellcolor{white} \phantom{a} & \cellcolor{gray!8}    ?v0 dbp:background ?v1 .  ?v0 foaf:name ?v2 .  ?v0 foaf:img ?v3 . & \cellcolor{gray!25} ?v0 dbp:birthPlace ?v1 .  ?v0 dbo:birthdate ?v2 .  ?v0 foaf:name ?v3 . \\
			\cellcolor{white} & \cellcolor{white} \phantom{a} & \cellcolor{gray!25}    ?v0 dbp:birthPlace ?v1 .  ?v0 dbp:birth ?v2 .  ?v0 foaf:name ?v3 . & \cellcolor{gray!8} ?v0 dbp:director ?v1 .  ?v0 dbo:budget ?v2 .  ?v0 dbp:runtime ?v3 . \\
			\cellcolor{white} & \cellcolor{white} \phantom{a} & \cellcolor{gray!8}    ?v0 dbp:genre ?v1 .  ?v0 foaf:name ?v2 .  ?v0 foaf:page ?v3 . & \cellcolor{gray!25} ?v0 dbp:gini ?v1 .  ?v0 dbp:giniYear ?v2 .  ?v0 a dbo:Country . \\
			\cellcolor{white} & \cellcolor{white} \phantom{a} & \cellcolor{gray!25}    ?v0 dbp:height ?v1 .  ?v0 dbp:wikiPageUsesTemplate ?v2 .  ?v0 dbp:locale ?v3 . & \cellcolor{gray!8} ?v0 dbp:leaderName ?v1 .  ?v0 dbp:subdivisionName ?v2 .  ?v0 dbp:elevation ?v3 . \\
			\cellcolor{white} & \cellcolor{white} \phantom{a} & \cellcolor{gray!8}    ?v0 dbp:location ?v1 .  ?v0 dbp:companyName ?v2 .  ?v0 skos:subject ?v3 . & \cellcolor{gray!25} ?v0 dbp:name ?v1 .  ?v0 dbo:foundationplace ?v2 .  ?v0 dbo:industry ?v3 . \\
			\cellcolor{white} & \cellcolor{white} \phantom{a} & \cellcolor{gray!25}    ?v0 dbp:starring ?v1 .  ?v0 dbp:director ?v1 .  ?v0 dbp:gross ?v2 . & \cellcolor{gray!8} ?v0 dbp:startDateProperty ?v1 .  ?v0 dbp:name ?v2 .  ?v0 dbo:foundationplace ?v3 . \\
			\cellcolor{white} & \cellcolor{white} \phantom{a} & \cellcolor{gray!8}    ?v0 dbp:subjectName ?v1 .  ?v0 geo:lat ?v2 .  ?v0 geo:long ?v3 . & \cellcolor{gray!25} ?v0 foaf:homepage ?v1 .  ?v0 foaf:name ?v2 .  ?v0 a yago:University108286163 . \\
			\cellcolor{white} & \cellcolor{white} \phantom{a} & \cellcolor{gray!25}    ?v0 foaf:name ?v1 .  ?v0 a foaf:Person .  ?v0 dbo:height ?v2 . & \cellcolor{gray!8} ?v0 foaf:name ?v1 .  ?v0 dbp:genre ?v2 .  ?v0 dbp:origin ?v3 . \\
			\cellcolor{white} & \cellcolor{white} \phantom{a} & \cellcolor{gray!8}    ?v0 foaf:name ?v1 .  ?v0 foaf:img ?v2 .  ?v0 dbp:abstract ?v3 . & \cellcolor{gray!25} ?v0 foaf:name ?v1 .  ?v0 foaf:img ?v2 .  ?v0 dbp:birth ?v3 . \\
			\cellcolor{white} & \cellcolor{white} \phantom{a} & \cellcolor{gray!25}    ?v0 foaf:name ?v1 .  ?v0 foaf:page ?v2 .  ?v0 foaf:depiction ?v3 . & \cellcolor{gray!8} ?v0 foaf:name ?v1 .  ?v0 rdfs:label ?v2 .  ?v0 dbp:abstract ?v3 . \\
			\cellcolor{white} & \cellcolor{white} \phantom{a} & \cellcolor{gray!8}    ?v0 foaf:name ?v1 .  ?v0 rdfs:label ?v2 .  ?v0 dbp:reference ?v3 . & \cellcolor{gray!25} ?v0 geo:lat ?v1 .  ?v0 geo:long ?v2 .  ?v0 foaf:page ?v3 . \\
			\cellcolor{white} & \cellcolor{white} \phantom{a} & \cellcolor{gray!25}    ?v0 geo:lat ?v1 .  ?v0 geo:long ?v2 .  ?v0 rdfs:label ?v3 . & \cellcolor{gray!8} ?v0 rdfs:comment ?v1 .  ?v0 rdfs:label ?v2 .  ?v0 dbp:award ?v3 . \\
			\cellcolor{white} & \cellcolor{white} \phantom{a} & \cellcolor{gray!8}    ?v0 rdfs:comment ?v1 .  ?v0 rdfs:label ?v2 .  ?v0 dbp:released ?v3 . & \cellcolor{gray!25} ?v0 rdfs:label ?v1 .  ?v0 a yago:Country108544813 .  ?v0 dbp:establishedDate ?v2 . \\
			\cellcolor{white} & \cellcolor{white} \phantom{a} & \cellcolor{gray!25}    ?v0 rdfs:label ?v1 .  ?v0 dbo:numberOfEmployees ?v2 .  ?v0 dbo:revenue ?v3 . & \cellcolor{gray!8} ?v0 rdfs:label ?v1 .  ?v0 dbp:presidentStart ?v2 .  ?v0 dbp:presidentEnd ?v3 . \\
			\cellcolor{white} & \cellcolor{white} \phantom{a} & \cellcolor{gray!8}    ?v0 rdfs:label ?v1 .  ?v0 foaf:name ?v2 .  ?v0 rdfs:comment ?v3 . & \cellcolor{gray!25} ?v0 rdfs:label ?v1 .  ?v0 rdfs:comment ?v2 .  ?v0 a dbo:City . \\
			\cellcolor{white} & \cellcolor{white} \phantom{a} & \cellcolor{gray!25}    ?v0 skos:subject ?v1 .  ?v0 foaf:name ?v2 .  ?v0 rdfs:comment ?v3 . & \cellcolor{gray!8} ?v0 skos:subject ?v1 .  ?v0 geo:lat ?v2 .  ?v0 geo:long ?v3 . \\
			\cellcolor{white} & \cellcolor{white} \phantom{a} & \cellcolor{gray!8}    ?v0 skos:subject ?v1 .  ?v0 rdfs:label ?v2 .  ?v0 rdfs:comment ?v3 . & \cellcolor{gray!25} \\
			\bottomrule
		\end{tabular}
	}
\end{table}

\begin{table}
	\centering
	\rowcolors{2}{gray!25}{gray!8}
	\resizebox{\textwidth}{!}{
		\begin{tabular}{!{\color{white}\vrule}l!{\color{white}\vrule}c@{\hs}!{\color{white}\vrule}p{11.5cm}!{\color{white}\vrule}p{11.5cm}!{\color{white}\vrule}}
			\toprule
			Category & \phantom{a} & \multicolumn{2}{c}{Query} \\
			\cmidrule{1-1} \cmidrule{3-4}

			\rowcolor{white}
			4 & \phantom{a} & \\
			\cellcolor{white} & \cellcolor{white} \phantom{a} & \cellcolor{gray!8}    ?v0 a dbo:Country .  ?v0 rdfs:label ?v1 .  ?v0 dbp:iso31661Alpha ?v2 .  ?v0 dbp:populationCensus ?v3 . & \cellcolor{gray!25} ?v0 a dbo:Country .  ?v0 rdfs:label ?v1 .  ?v0 dbp:latd ?v2 .  ?v0 dbp:longd ?v3 . \\
			\cellcolor{white} & \cellcolor{white} \phantom{a} & \cellcolor{gray!25}    ?v0 a dbo:Mountain .  ?v0 dbo:elevation ?v1 .  ?v0 dbp:name ?v2 .  ?v0 dbp:range ?v3 . & \cellcolor{gray!8} ?v0 a dbo:Person .  ?v0 dbo:birthdate ?v1 .  ?v0 foaf:name ?v2 .  ?v0 dbp:abstract ?v3 . \\
			\cellcolor{white} & \cellcolor{white} \phantom{a} & \cellcolor{gray!8}    ?v0 dbo:birthdate ?v1 .  ?v0 dbo:birthplace ?v2 .  ?v0 foaf:name ?v3 .  ?v0 dbo:deathdate ?v4 . & \cellcolor{gray!25} ?v0 dbo:birthdate ?v1 .  ?v0 foaf:name ?v2 .  ?v0 dbo:deathdate ?v3 .  ?v0 dbp:knownFor ?v4 . \\
			\cellcolor{white} & \cellcolor{white} \phantom{a} & \cellcolor{gray!25}    ?v0 dbo:birthplace ?v1 .  ?v0 dbo:birthdate ?v2 .  ?v0 foaf:name ?v3 .  ?v0 rdfs:comment ?v4 . & \cellcolor{gray!8} ?v0 dbo:birthplace ?v1 .  ?v0 skos:subject ?v2 .  ?v0 dbo:birthdate ?v3 .  ?v0 foaf:name ?v4 . \\
			\cellcolor{white} & \cellcolor{white} \phantom{a} & \cellcolor{gray!8}    ?v0 dbo:currentTeam ?v1 .  ?v0 dbo:birthdate ?v2 .  ?v0 foaf:name ?v3 .  ?v0 rdfs:comment ?v4 . & \cellcolor{gray!25} ?v0 dbo:influences ?v1 .  ?v0 foaf:page ?v2 .  ?v0 rdfs:label ?v3 .  ?v0 foaf:name ?v4 . \\
			\cellcolor{white} & \cellcolor{white} \phantom{a} & \cellcolor{gray!25}    ?v0 dbo:series ?v1 .  ?v0 dbo:releaseDate ?v2 .  ?v0 dbo:episodeNumber ?v3 .  ?v0 dbo:seasonNumber ?v4 . & \cellcolor{gray!8} ?v0 dbp:abstract ?v1 .  ?v0 geo:lat ?v2 .  ?v0 geo:long ?v3 .  ?v0 foaf:page ?v4 . \\
			\cellcolor{white} & \cellcolor{white} \phantom{a} & \cellcolor{gray!8}    ?v0 dbp:background ?v1 .  ?v0 foaf:name ?v2 .  ?v0 dbo:birthplace ?v3 .  ?v0 foaf:img ?v4 . & \cellcolor{gray!25} ?v0 dbp:birthPlace ?v1 .  ?v0 dbo:birthdate ?v2 .  ?v0 foaf:name ?v3 .  ?v0 dbo:deathdate ?v4 . \\
			\cellcolor{white} & \cellcolor{white} \phantom{a} & \cellcolor{gray!25}    ?v0 dbp:birthPlace ?v1 .  ?v0 dbo:birthdate ?v2 .  ?v0 foaf:name ?v3 .  ?v0 dbp:ocupation ?v4 . & \cellcolor{gray!8} ?v0 dbp:birthPlace ?v1 .  ?v0 dbp:birth ?v2 .  ?v0 dbp:death ?v3 .  ?v0 foaf:name ?v4 . \\
			\cellcolor{white} & \cellcolor{white} \phantom{a} & \cellcolor{gray!8}    ?v0 dbp:height ?v1 .  ?v0 dbp:wikiPageUsesTemplate ?v2 .  ?v0 dbp:locale ?v3 .  ?v0 dbp:began ?v4 . & \cellcolor{gray!25} ?v0 foaf:homepage ?v1 .  ?v0 foaf:name ?v2 .  ?v0 foaf:img ?v3 .  ?v0 a yago:University108286163 . \\
			\cellcolor{white} & \cellcolor{white} \phantom{a} & \cellcolor{gray!25}    ?v0 foaf:name ?v1 .  ?v0 foaf:depiction ?v2 .  ?v0 rdfs:label ?v3 .  ?v0 dbp:abstract ?v4 . & \cellcolor{gray!8} ?v0 foaf:name ?v1 .  ?v0 foaf:img ?v2 .  ?v0 dbp:abstract ?v3 .  ?v0 a yago:University108286163 . \\
			\cellcolor{white} & \cellcolor{white} \phantom{a} & \cellcolor{gray!8}    ?v0 foaf:name ?v1 .  ?v0 rdfs:label ?v2 .  ?v0 dbp:reference ?v3 .  ?v0 dbp:abstract ?v4 . & \cellcolor{gray!25} ?v0 rdfs:label ?v1 .  ?v0 dbo:numberOfEmployees ?v2 .  ?v0 dbo:revenue ?v3 .  ?v0 dbp:foundation ?v4 . \\
			\cellcolor{white} & \cellcolor{white} \phantom{a} & \cellcolor{gray!25}    ?v0 rdfs:label ?v1 .  ?v0 dbp:occupation ?v2 .  ?v0 dbp:birthPlace ?v3 .  ?v0 foaf:img ?v4 . & \cellcolor{gray!8} ?v0 rdfs:label ?v1 .  ?v0 dbp:populationMetro ?v2 .  ?v0 dbp:areaTotalKm ?v3 .  ?v0 dbp:subdivisionName ?v4 . \\
			\cellcolor{white} & \cellcolor{white} \phantom{a} & \cellcolor{gray!8}    ?v0 skos:subject ?v1 .  ?v0 dbo:birthdate ?v2 .  ?v0 foaf:name ?v3 .  ?v0 rdfs:comment ?v4 . & \cellcolor{gray!25} ?v0 skos:subject ?v1 .  ?v0 foaf:name ?v2 .  ?v0 foaf:img ?v3 .  ?v0 dbp:abstract ?v4 . \\
			
			\midrule
			\rowcolor{white}
			9 & \phantom{a} & \\
			\cellcolor{white} & \cellcolor{white} \phantom{a} & \multicolumn{2}{p{23.45cm}}{  ?v0 dbo:binomial\_authority ?v1 .  ?v0 dbo:classis ?v2 .  ?v0 dbo:division ?v3 .  ?v0 dbo:family ?v4 .  ?v0 dbo:genus ?v5 .  ?v0 dbo:kingdom ?v6 .  ?v0 dbo:order ?v7 .  ?v0 dbo:species ?v8 .  ?v0 foaf:img ?v9 . } \\

			\midrule
			\rowcolor{white}
			10 & \phantom{a} & \\
			\cellcolor{white} & \cellcolor{white} \phantom{a} & \multicolumn{2}{p{23.45cm}}{?v0 dbo:binomial\_authority ?v1 .  ?v0 dbo:classis ?v2 .  ?v0 dbo:division ?v3 .  ?v0 dbo:family ?v4 .  ?v0 dbo:genus ?v5 .  ?v0 dbo:kingdom ?v6 .  ?v0 dbo:order ?v7 .  ?v0 dbo:species ?v8 .  ?v0 foaf:img ?v9 .  ?v0 rdfs:label ?v10 .} \\
			\bottomrule
		\end{tabular}
	}
\end{table}
