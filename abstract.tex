The advent of the Internet enabled the sharing of information between people all around the world. Projects like \emph{Wikipedia} have made human knowledge accessible to anybody with a simple mouse click. The Linked Data movement has made a considerable leap in the amount of data now available on the Web. Data about science, social interactions, governments, entertainment and news is now available to anybody.

That data is described at a very fine granularity, allowing to describe precisely entities (people, films, monuments, \ldots) and relationships between entities. This marks a shift in data management on the Web: instead of a graph of web documents, we witness now a graph of entities with links carrying semantic; we call this Web Data.\\

Web Data is characterized by the use of the Resource Description Framework (RDF) data model. The RDF model enables a flexible and dynamic management of information, allowing anyone to easily create and modify data. In the presence of such a flexible data model, the amount of information grew organically: the structure of the data is not necessarily maintained over time, and some data may be created by integrating several existing datasources. Therefore, this data is referred to as \emph{semi-structured}.\\

The result is that Web Data is a large collection of semi-structured and heterogeneous data. It is then difficult for a user to understand what information is contained in a particular dataset within that collection or how to access it. In this thesis, we propose the use of \emph{graph summarisation} for highlighting the structure of a dataset. The output of the summarisation is generated from the data itself and is akin to a schema, which we assume is at the core of many applications such as query optimisation, data exploration, and data integration, \ldots.
We present in this thesis a formal model for graph summarisation and describe how to generate the \emph{summary} of a dataset. Graph summarisation is a technique applied directly on the data, and as such it is susceptible to the quality of the data. Hence, we introduce a model for assessing the precision of a summary with regards to the data. Finally, we develop several applications that leverage the summary of datasets.
