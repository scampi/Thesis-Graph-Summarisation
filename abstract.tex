The advent of the Internet enabled the sharing of information between people all around the world. Projects like \emph{Wikipedia} have made human knowledge accessible to anybody with a simple mouse click. The Linked Data movement has made a considerable leap in the amount of data now available on the Web. Data about science, social interactions, governments, entertainment and news is now available to anybody.

That data is described at a very fine granularity, allowing to describe precisely entities (people, films, monuments, \ldots) and relationships between entities. This marks a shift in data management on the Web: instead of a graph of web documents, we witness now a graph of entities with links carrying semantic; we call this \emph{Web Data}.\\

Web Data is characterized by the use of the Resource Description Framework data model. This model enables a flexible and dynamic management of information, allowing anyone to easily create and modify data. In the presence of such a flexible data model, the amount of information grew organically: the structure of the data is not necessarily maintained over time, and some data may be created by integrating several existing datasources, impacting on the structure consistency of the resulting integration. Therefore, this data is referred to as \emph{semi-structured}.\\

The result is that Web Data is a large collection of semi-structured and heterogeneous data. It is then difficult for a user to understand what information is contained in a particular dataset within that collection or how to access it. In relational database systems, the schema answers to that need; which in the context of Web Data is generally missing.\\

In this thesis, we propose the use of \emph{graph summarisation} for highlighting the structure of a dataset, which output is referred to as the \emph{graph summary}. The graph summary is a graph, generated from the data itself, which shares the same structure as the original dataset. Unlike the original graph, the summary abstracts itself from details about entities and is focused on the structure of the graph. Therefore, the graph summary is akin to a schema, which we assume is at the core of many applications such as query optimisation, data exploration, data integration, \ldots. The generation of a graph summary is then intended to be at the core of such applications within the context of Web Data.\\

We present in this thesis a formal model for graph summarisation and describe how to generate the \emph{summary} of a dataset. Graph summarisation is a technique applied directly on the data, and as such it is susceptible to the quality of the data. Hence, we introduce a model for assessing the precision of a summary with regards to the data. Finally, we develop several applications that leverage the summary of datasets.
