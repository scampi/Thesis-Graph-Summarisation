\chapter{Graph Summary: a Novel Schema for Web Data}
\label{chap:summary}

The Web of Data consists of a tremendous amount of structured datasets, coming from a variety of sources. An attractive aspect of the Web of Data is the total freedom with regards to the production of ``knowledge''; as stated by Berners-Lee in~\cite{tbl:1997:wam}, ``Anyone can say anything about anything''. Undoubtedly, this played a role in the considerable growth of the Web of Data. However, this also means that there is no one enforcing some \emph{quality} about the data, e.g., what design structure to use, which vocabulary terms, checking data consistency.

Although there exists a number of ontologies for describing a wealth of data, datasets do not follow strictly the specifications. Enforcing the structure would stand against the design choice of the Web of Data stated above. An other approach is to go with the flow and \emph{generate} such specifications directly from the data itself.

Graph summarisation is a technique that maps a graph into another, \emph{smaller} graph. That new graph keeps the same \emph{structure} as the original graph, but contains less nodes and less edges. The new graph defines then a \emph{schema} of the graph. In this chapter, we present how to summarise graphs in the context of Web of Data.
