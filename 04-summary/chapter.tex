\chapter{Graph Summary: a Novel Schema for Web Data}
\label{chap:summary}

The Web Data consists of a tremendous amount of structured datasets, coming from a variety of sources. An attractive aspect of the Web Data is the total freedom with regards to the production of ``knowledge''; as stated by Berners-Lee in~\cite{tbl:1997:wam}, ``Anyone can say anything about anything''. Undoubtedly, the facility to publish data played a role in the considerable growth of the Web Data. However, this also means that there is nothing enforcing some \emph{quality} of the data, e.g., what design structure or vocabulary terms to use, or checking data consistency.

Although there exists a number of ontologies for describing a wealth of data, datasets do not follow strictly the specifications. Enforcing the structure would stand against the design choice of the Web Data stated above. Another approach is to go with the flow and \emph{generate} such specifications directly from the data itself.

Graph summarisation is a technique that maps an entity graph into another, \emph{smaller} graph. That new graph keeps the same \emph{structure} as the entity graph, but contains less nodes and less edges. % The new graph defines then a \emph{schema} of the graph.
As we aim to highlight the underlying structure of a graph, we consider already existing features for the summarisation, built alongside the graph. Such features are for example the \gls{attributes} and \gls{types} of a graph.
In particular, data exhibiting graph heterogeneity with regards to the structure or the vocabulary can be summarised effectively with the presented methods. It is important to note that the notion of data quality indicates in general errors or mistakes in the data. This is not a concern within this thesis as data quality is orthogonal to the notion of heterogeneity.

In this chapter, we present a generic framework for summarising graphs in the context of Web Data, generating either precise or approximate summaries. In addition, we describe possible implementations of the framework.
