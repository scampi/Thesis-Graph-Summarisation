\section{Conclusion and Future Work}

In this chapter, we introduced ``MF'', which is a novel ranking model for semi-structured data. We showed that MF is a generalisation of field-based ranking models to a DAG model. This allows to apply the MF ranking model to a variety of data modelling: tree-shaped data as discussed in Section~\ref{sec:experiments}, or graph-shaped data as presented in Section~\ref{sec:summary-ranking:eval}.

The MF ranking model introduces two levels of normalisations; one over the length of a node, the other over the degree of a node. The latter is a new normalisation level when compared to field-based ranking models. Through a variety of experiments, these two levels have proved to increase the ranking performance significantly, since they allow a finer tuning of the ranking with regards to the data.\\

We have shown in the evaluations that MF ranking normalisation parameters are highly dependent on the datasets. MF ranking on a highly curated dataset requires different parameters than on a heterogeneous one. As future work, we will then pursue two directions of research. First, a dynamic approach for finding such parameters would improve the performance of the approach. Second, we will investigate the possibility of integrating statistics from a graph summary into the ranking, e.g., as a weight for an edge.
